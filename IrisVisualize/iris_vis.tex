% Options for packages loaded elsewhere
\PassOptionsToPackage{unicode}{hyperref}
\PassOptionsToPackage{hyphens}{url}
%
\documentclass[
]{article}
\usepackage{lmodern}
\usepackage{amssymb,amsmath}
\usepackage{ifxetex,ifluatex}
\ifnum 0\ifxetex 1\fi\ifluatex 1\fi=0 % if pdftex
  \usepackage[T1]{fontenc}
  \usepackage[utf8]{inputenc}
  \usepackage{textcomp} % provide euro and other symbols
\else % if luatex or xetex
  \usepackage{unicode-math}
  \defaultfontfeatures{Scale=MatchLowercase}
  \defaultfontfeatures[\rmfamily]{Ligatures=TeX,Scale=1}
\fi
% Use upquote if available, for straight quotes in verbatim environments
\IfFileExists{upquote.sty}{\usepackage{upquote}}{}
\IfFileExists{microtype.sty}{% use microtype if available
  \usepackage[]{microtype}
  \UseMicrotypeSet[protrusion]{basicmath} % disable protrusion for tt fonts
}{}
\makeatletter
\@ifundefined{KOMAClassName}{% if non-KOMA class
  \IfFileExists{parskip.sty}{%
    \usepackage{parskip}
  }{% else
    \setlength{\parindent}{0pt}
    \setlength{\parskip}{6pt plus 2pt minus 1pt}}
}{% if KOMA class
  \KOMAoptions{parskip=half}}
\makeatother
\usepackage{xcolor}
\IfFileExists{xurl.sty}{\usepackage{xurl}}{} % add URL line breaks if available
\IfFileExists{bookmark.sty}{\usepackage{bookmark}}{\usepackage{hyperref}}
\hypersetup{
  pdftitle={Iris: Data Manipulation + Visualization},
  hidelinks,
  pdfcreator={LaTeX via pandoc}}
\urlstyle{same} % disable monospaced font for URLs
\usepackage[margin=1in]{geometry}
\usepackage{color}
\usepackage{fancyvrb}
\newcommand{\VerbBar}{|}
\newcommand{\VERB}{\Verb[commandchars=\\\{\}]}
\DefineVerbatimEnvironment{Highlighting}{Verbatim}{commandchars=\\\{\}}
% Add ',fontsize=\small' for more characters per line
\usepackage{framed}
\definecolor{shadecolor}{RGB}{248,248,248}
\newenvironment{Shaded}{\begin{snugshade}}{\end{snugshade}}
\newcommand{\AlertTok}[1]{\textcolor[rgb]{0.94,0.16,0.16}{#1}}
\newcommand{\AnnotationTok}[1]{\textcolor[rgb]{0.56,0.35,0.01}{\textbf{\textit{#1}}}}
\newcommand{\AttributeTok}[1]{\textcolor[rgb]{0.77,0.63,0.00}{#1}}
\newcommand{\BaseNTok}[1]{\textcolor[rgb]{0.00,0.00,0.81}{#1}}
\newcommand{\BuiltInTok}[1]{#1}
\newcommand{\CharTok}[1]{\textcolor[rgb]{0.31,0.60,0.02}{#1}}
\newcommand{\CommentTok}[1]{\textcolor[rgb]{0.56,0.35,0.01}{\textit{#1}}}
\newcommand{\CommentVarTok}[1]{\textcolor[rgb]{0.56,0.35,0.01}{\textbf{\textit{#1}}}}
\newcommand{\ConstantTok}[1]{\textcolor[rgb]{0.00,0.00,0.00}{#1}}
\newcommand{\ControlFlowTok}[1]{\textcolor[rgb]{0.13,0.29,0.53}{\textbf{#1}}}
\newcommand{\DataTypeTok}[1]{\textcolor[rgb]{0.13,0.29,0.53}{#1}}
\newcommand{\DecValTok}[1]{\textcolor[rgb]{0.00,0.00,0.81}{#1}}
\newcommand{\DocumentationTok}[1]{\textcolor[rgb]{0.56,0.35,0.01}{\textbf{\textit{#1}}}}
\newcommand{\ErrorTok}[1]{\textcolor[rgb]{0.64,0.00,0.00}{\textbf{#1}}}
\newcommand{\ExtensionTok}[1]{#1}
\newcommand{\FloatTok}[1]{\textcolor[rgb]{0.00,0.00,0.81}{#1}}
\newcommand{\FunctionTok}[1]{\textcolor[rgb]{0.00,0.00,0.00}{#1}}
\newcommand{\ImportTok}[1]{#1}
\newcommand{\InformationTok}[1]{\textcolor[rgb]{0.56,0.35,0.01}{\textbf{\textit{#1}}}}
\newcommand{\KeywordTok}[1]{\textcolor[rgb]{0.13,0.29,0.53}{\textbf{#1}}}
\newcommand{\NormalTok}[1]{#1}
\newcommand{\OperatorTok}[1]{\textcolor[rgb]{0.81,0.36,0.00}{\textbf{#1}}}
\newcommand{\OtherTok}[1]{\textcolor[rgb]{0.56,0.35,0.01}{#1}}
\newcommand{\PreprocessorTok}[1]{\textcolor[rgb]{0.56,0.35,0.01}{\textit{#1}}}
\newcommand{\RegionMarkerTok}[1]{#1}
\newcommand{\SpecialCharTok}[1]{\textcolor[rgb]{0.00,0.00,0.00}{#1}}
\newcommand{\SpecialStringTok}[1]{\textcolor[rgb]{0.31,0.60,0.02}{#1}}
\newcommand{\StringTok}[1]{\textcolor[rgb]{0.31,0.60,0.02}{#1}}
\newcommand{\VariableTok}[1]{\textcolor[rgb]{0.00,0.00,0.00}{#1}}
\newcommand{\VerbatimStringTok}[1]{\textcolor[rgb]{0.31,0.60,0.02}{#1}}
\newcommand{\WarningTok}[1]{\textcolor[rgb]{0.56,0.35,0.01}{\textbf{\textit{#1}}}}
\usepackage{graphicx,grffile}
\makeatletter
\def\maxwidth{\ifdim\Gin@nat@width>\linewidth\linewidth\else\Gin@nat@width\fi}
\def\maxheight{\ifdim\Gin@nat@height>\textheight\textheight\else\Gin@nat@height\fi}
\makeatother
% Scale images if necessary, so that they will not overflow the page
% margins by default, and it is still possible to overwrite the defaults
% using explicit options in \includegraphics[width, height, ...]{}
\setkeys{Gin}{width=\maxwidth,height=\maxheight,keepaspectratio}
% Set default figure placement to htbp
\makeatletter
\def\fps@figure{htbp}
\makeatother
\setlength{\emergencystretch}{3em} % prevent overfull lines
\providecommand{\tightlist}{%
  \setlength{\itemsep}{0pt}\setlength{\parskip}{0pt}}
\setcounter{secnumdepth}{-\maxdimen} % remove section numbering

\title{Iris: Data Manipulation + Visualization}
\author{}
\date{\vspace{-2.5em}}

\begin{document}
\maketitle

Edgar Anderson's Iris Data Description This famous (Fisher's or
Anderson's) iris data set gives the measurements in centimeters of the
variables sepal length and width and petal length and width,
respectively, for 50 flowers from each of 3 species of iris. The species
are Iris setosa, versicolor, and virginica.

\begin{Shaded}
\begin{Highlighting}[]
\KeywordTok{head}\NormalTok{(iris)}
\end{Highlighting}
\end{Shaded}

\begin{verbatim}
##   Sepal.Length Sepal.Width Petal.Length Petal.Width Species
## 1          5.1         3.5          1.4         0.2  setosa
## 2          4.9         3.0          1.4         0.2  setosa
## 3          4.7         3.2          1.3         0.2  setosa
## 4          4.6         3.1          1.5         0.2  setosa
## 5          5.0         3.6          1.4         0.2  setosa
## 6          5.4         3.9          1.7         0.4  setosa
\end{verbatim}

\begin{Shaded}
\begin{Highlighting}[]
\KeywordTok{str}\NormalTok{(iris)}
\end{Highlighting}
\end{Shaded}

\begin{verbatim}
## 'data.frame':    150 obs. of  5 variables:
##  $ Sepal.Length: num  5.1 4.9 4.7 4.6 5 5.4 4.6 5 4.4 4.9 ...
##  $ Sepal.Width : num  3.5 3 3.2 3.1 3.6 3.9 3.4 3.4 2.9 3.1 ...
##  $ Petal.Length: num  1.4 1.4 1.3 1.5 1.4 1.7 1.4 1.5 1.4 1.5 ...
##  $ Petal.Width : num  0.2 0.2 0.2 0.2 0.2 0.4 0.3 0.2 0.2 0.1 ...
##  $ Species     : Factor w/ 3 levels "setosa","versicolor",..: 1 1 1 1 1 1 1 1 1 1 ...
\end{verbatim}

\begin{Shaded}
\begin{Highlighting}[]
\CommentTok{#libraries:}
\KeywordTok{library}\NormalTok{(MASS)}
\KeywordTok{library}\NormalTok{(dplyr)}
\end{Highlighting}
\end{Shaded}

\begin{verbatim}
## Warning: package 'dplyr' was built under R version 4.0.3
\end{verbatim}

\begin{verbatim}
## 
## Attaching package: 'dplyr'
\end{verbatim}

\begin{verbatim}
## The following object is masked from 'package:MASS':
## 
##     select
\end{verbatim}

\begin{verbatim}
## The following objects are masked from 'package:stats':
## 
##     filter, lag
\end{verbatim}

\begin{verbatim}
## The following objects are masked from 'package:base':
## 
##     intersect, setdiff, setequal, union
\end{verbatim}

\begin{Shaded}
\begin{Highlighting}[]
\KeywordTok{library}\NormalTok{(ggplot2)}
\end{Highlighting}
\end{Shaded}

\begin{verbatim}
## Warning: package 'ggplot2' was built under R version 4.0.3
\end{verbatim}

\begin{Shaded}
\begin{Highlighting}[]
\KeywordTok{library}\NormalTok{(magrittr) }\CommentTok{#%>% operator}
\end{Highlighting}
\end{Shaded}

\begin{verbatim}
## Warning: package 'magrittr' was built under R version 4.0.3
\end{verbatim}

\begin{Shaded}
\begin{Highlighting}[]
\KeywordTok{library}\NormalTok{(tidyr)}
\end{Highlighting}
\end{Shaded}

\begin{verbatim}
## Warning: package 'tidyr' was built under R version 4.0.3
\end{verbatim}

\begin{verbatim}
## 
## Attaching package: 'tidyr'
\end{verbatim}

\begin{verbatim}
## The following object is masked from 'package:magrittr':
## 
##     extract
\end{verbatim}

iris is a data frame with 150 cases (rows) and 5 variables (columns)
named Sepal.Length, Sepal.Width, Petal.Length, Petal.Width, and Species.
We rearrange columns for better aesthetics of the data, and it would be
helpful when we tidy data.

\begin{Shaded}
\begin{Highlighting}[]
\KeywordTok{str}\NormalTok{(iris)}
\end{Highlighting}
\end{Shaded}

\begin{verbatim}
## 'data.frame':    150 obs. of  5 variables:
##  $ Sepal.Length: num  5.1 4.9 4.7 4.6 5 5.4 4.6 5 4.4 4.9 ...
##  $ Sepal.Width : num  3.5 3 3.2 3.1 3.6 3.9 3.4 3.4 2.9 3.1 ...
##  $ Petal.Length: num  1.4 1.4 1.3 1.5 1.4 1.7 1.4 1.5 1.4 1.5 ...
##  $ Petal.Width : num  0.2 0.2 0.2 0.2 0.2 0.4 0.3 0.2 0.2 0.1 ...
##  $ Species     : Factor w/ 3 levels "setosa","versicolor",..: 1 1 1 1 1 1 1 1 1 1 ...
\end{verbatim}

\begin{Shaded}
\begin{Highlighting}[]
\CommentTok{#rearranging columns}
\NormalTok{iris1<-iris[}\KeywordTok{c}\NormalTok{(}\DecValTok{5}\NormalTok{,}\DecValTok{1}\OperatorTok{:}\DecValTok{4}\NormalTok{)]}
\KeywordTok{str}\NormalTok{(iris1)}
\end{Highlighting}
\end{Shaded}

\begin{verbatim}
## 'data.frame':    150 obs. of  5 variables:
##  $ Species     : Factor w/ 3 levels "setosa","versicolor",..: 1 1 1 1 1 1 1 1 1 1 ...
##  $ Sepal.Length: num  5.1 4.9 4.7 4.6 5 5.4 4.6 5 4.4 4.9 ...
##  $ Sepal.Width : num  3.5 3 3.2 3.1 3.6 3.9 3.4 3.4 2.9 3.1 ...
##  $ Petal.Length: num  1.4 1.4 1.3 1.5 1.4 1.7 1.4 1.5 1.4 1.5 ...
##  $ Petal.Width : num  0.2 0.2 0.2 0.2 0.2 0.4 0.3 0.2 0.2 0.1 ...
\end{verbatim}

The principles of tidy data provide a standard way to organise data
values within a dataset. A standard makes initial data cleaning easier
because you don't need to start from scratch and reinvent the wheel
every time. The tidy data standard has been designed to facilitate
initial exploration and analysis of the data, and to simplify the
development of data analysis tools that work well together. In R, we can
use the library: tidyr for data cleaning,

\begin{Shaded}
\begin{Highlighting}[]
\KeywordTok{library}\NormalTok{(tidyr)}
\CommentTok{#tidying data using tidyr}
\NormalTok{iris.tidy <-}\StringTok{ }\NormalTok{iris1 }\OperatorTok
\StringTok{  }\KeywordTok{gather}\NormalTok{(key, Value, }\OperatorTok{-}\NormalTok{Species) }\OperatorTok
\StringTok{  }\KeywordTok{separate}\NormalTok{(key, }\KeywordTok{c}\NormalTok{(}\StringTok{"Part"}\NormalTok{, }\StringTok{"Measure"}\NormalTok{), }\StringTok{"}\CharTok{\textbackslash{}\textbackslash{}}\StringTok{."}\NormalTok{)}
\KeywordTok{str}\NormalTok{(iris.tidy)}
\end{Highlighting}
\end{Shaded}

\begin{verbatim}
## 'data.frame':    600 obs. of  4 variables:
##  $ Species: Factor w/ 3 levels "setosa","versicolor",..: 1 1 1 1 1 1 1 1 1 1 ...
##  $ Part   : chr  "Sepal" "Sepal" "Sepal" "Sepal" ...
##  $ Measure: chr  "Length" "Length" "Length" "Length" ...
##  $ Value  : num  5.1 4.9 4.7 4.6 5 5.4 4.6 5 4.4 4.9 ...
\end{verbatim}

This is just a small instance of data tidying that I've used to skip the
manual steps it takes to tidy data using data manipulation functions
provided in the library: dplyr. For beginners, I have shown how data
manipulation can be done manually in the following code chunk.

\begin{Shaded}
\begin{Highlighting}[]
\KeywordTok{library}\NormalTok{(dplyr)}
\KeywordTok{library}\NormalTok{(magrittr)}

\CommentTok{#import data}
\NormalTok{iris<-}\KeywordTok{read.csv}\NormalTok{(}\StringTok{"datasets/iris.csv"}\NormalTok{)}
\CommentTok{#Separating sepal data}
\NormalTok{iris1a<-iris }\OperatorTok
\StringTok{  }\KeywordTok{mutate}\NormalTok{(}\DataTypeTok{Part=}\StringTok{"Sepal"}\NormalTok{) }\OperatorTok
\StringTok{  }\KeywordTok{select}\NormalTok{(Species,Part,Sepal.Length,Sepal.Width) }\OperatorTok
\StringTok{  }\KeywordTok{rename}\NormalTok{(}\DataTypeTok{Length=}\NormalTok{Sepal.Length,}\DataTypeTok{Width=}\NormalTok{Sepal.Width)}
\CommentTok{#Separating petal data}
\NormalTok{iris1b<-iris }\OperatorTok
\StringTok{  }\KeywordTok{mutate}\NormalTok{(}\DataTypeTok{Part=}\StringTok{"Petal"}\NormalTok{) }\OperatorTok
\StringTok{  }\KeywordTok{select}\NormalTok{(Species,Part,Petal.Length,Petal.Width) }\OperatorTok
\StringTok{  }\KeywordTok{rename}\NormalTok{(}\DataTypeTok{Length=}\NormalTok{Petal.Length,}\DataTypeTok{Width=}\NormalTok{Petal.Width)}
\CommentTok{#Combining sepal and petal data}
\NormalTok{iris2<-}\KeywordTok{rbind}\NormalTok{(iris1a,iris1b)}
\NormalTok{iris2a<-iris2 }\OperatorTok
\StringTok{  }\KeywordTok{mutate}\NormalTok{(}\DataTypeTok{Measure=}\StringTok{"Length"}\NormalTok{) }\OperatorTok
\StringTok{  }\KeywordTok{select}\NormalTok{(Species,Part,Measure,Length) }\OperatorTok
\StringTok{  }\KeywordTok{rename}\NormalTok{(}\DataTypeTok{Value=}\NormalTok{Length)}
\NormalTok{iris2b<-iris2 }\OperatorTok
\StringTok{  }\KeywordTok{mutate}\NormalTok{(}\DataTypeTok{Measure=}\StringTok{"Width"}\NormalTok{) }\OperatorTok
\StringTok{  }\KeywordTok{select}\NormalTok{(Species,Part,Measure,Width) }\OperatorTok
\StringTok{  }\KeywordTok{rename}\NormalTok{(}\DataTypeTok{Value=}\NormalTok{Width)}
\NormalTok{iris3<-}\KeywordTok{rbind}\NormalTok{(iris2a,iris2b)}
\end{Highlighting}
\end{Shaded}

Now, we are at the step where we visualize the data we cleaned. The data
interpretation can be difficult in the following graph as we can't
really interpret anything useful.

\begin{Shaded}
\begin{Highlighting}[]
\KeywordTok{ggplot}\NormalTok{(}\DataTypeTok{data =}\NormalTok{ iris,}\KeywordTok{aes}\NormalTok{(}\DataTypeTok{x=}\NormalTok{Sepal.Length,}\DataTypeTok{y=}\NormalTok{Sepal.Width,}\DataTypeTok{col=}\NormalTok{Species))}\OperatorTok{+}
\StringTok{  }\KeywordTok{geom_jitter}\NormalTok{()}
\end{Highlighting}
\end{Shaded}

\includegraphics{iris_vis_files/figure-latex/unnamed-chunk-5-1.pdf}

The second graph is more visually pleasing as there isn't a mix up of
different species and we can even conclude a strong correlation for
measure of Petals depending on the species.

\begin{Shaded}
\begin{Highlighting}[]
\KeywordTok{ggplot}\NormalTok{(}\DataTypeTok{data =}\NormalTok{ iris,}\KeywordTok{aes}\NormalTok{(}\DataTypeTok{x=}\NormalTok{Petal.Length,}\DataTypeTok{y=}\NormalTok{Petal.Width,}\DataTypeTok{col=}\NormalTok{Species))}\OperatorTok{+}
\StringTok{  }\KeywordTok{geom_jitter}\NormalTok{()}
\end{Highlighting}
\end{Shaded}

\includegraphics{iris_vis_files/figure-latex/unnamed-chunk-6-1.pdf}

The plots in the graphs aren't accurate because we used the function
geom\_jitter Funtion geom\_point() maps the data as it is, but
geom\_jitter() adds a little variance to data for better representation
in several cases and avoid overlapping of same values.

In the next graph we are going to use a function facet\_grid() which
would make the visualization experience more aesthetic and easier to
interpret.

\begin{Shaded}
\begin{Highlighting}[]
\KeywordTok{ggplot}\NormalTok{(}\DataTypeTok{data=}\NormalTok{iris.tidy,}\KeywordTok{aes}\NormalTok{(}\DataTypeTok{x=}\NormalTok{Measure,}\DataTypeTok{y=}\NormalTok{Value,}\DataTypeTok{col=}\NormalTok{Species))}\OperatorTok{+}
\StringTok{  }\KeywordTok{geom_jitter}\NormalTok{()}\OperatorTok{+}
\StringTok{  }\KeywordTok{facet_grid}\NormalTok{(.}\OperatorTok{~}\NormalTok{Part)}
\end{Highlighting}
\end{Shaded}

\includegraphics{iris_vis_files/figure-latex/unnamed-chunk-7-1.pdf}

\end{document}
